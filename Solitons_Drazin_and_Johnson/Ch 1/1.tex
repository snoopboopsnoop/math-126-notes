\documentclass[12pt]{article}

%packages for math formatting and symbols
\usepackage{amssymb}
\usepackage{amsmath}
\usepackage{amsthm}
\usepackage{graphicx}
\graphicspath{
	{C:/Users/cheng/Documents/LaTeX/}
}
\usepackage{float}
\usepackage{hyperref}
\hypersetup {
	colorlinks=true,
}
\usepackage[breakable, theorems, skins]{tcolorbox}


\def\rcurs{{\mbox{$\resizebox{.16in}{.08in}{\includegraphics{ScriptR}}$}}}
\def\brcurs{{\mbox{$\resizebox{.16in}{.08in}{\includegraphics{BoldR}}$}}}
\def\hrcurs{{\mbox{$\hat \brcurs$}}}

%global lengths/margins
\allowdisplaybreaks
\unitlength 1cm
\textheight 22cm
\textwidth 17cm
\oddsidemargin -0.5cm
\evensidemargin -0.5cm
\topmargin -1.5cm
\topskip 0cm
\headheight 0.5cm
\headsep 1cm
%\marginparwidth 1.2cm
\newlength\doubleind
\addtolength{\doubleind}{\leftmargini}
\addtolength{\doubleind}{\leftmarginii}
\parindent 0pt
\def\lstsp{\hspace{\labelsep}}
\def\tab{\indent\hspace{15pt}}

%new commands
\def\dx{\mathrm{d}x}
\def\Z{\mathbb{Z}}
\def\N{\mathbb{N}}
\def\R{\mathbb{R}}
\def\Q{\mathbb{Q}}
\def\C{\mathbb{C}}
\def\qedsymbol{$\blacksquare$}

%calculus
\def\D{\mathrm{d}}
\def\dint{\displaystyle\int}
\newcommand{\diff}[3][]{\ensuremath{\frac{\D^{#1} #2}{\D #3^{#1}}}}
\newcommand{\partials}[3][]{\ensuremath{\frac{\partial^{#1} {#2}}{\partial {#3}^{#1}}}}

%theorems
\newtheorem{theorem}{Theorem}
\newtheorem*{theorem*}{Theorem}
\newtheorem{lemma}{Lemma}
\newtheorem*{lemma*}{Lemma}
\newtheorem{prop}{Proposition}
\newtheorem*{prop*}{Proposition}

\tcbset{
	exstyle/.style={enhanced, breakable, beforeafter skip balanced=10pt, coltitle=black, theorem style=plain, terminator sign={.\ \ \ }, fonttitle=\bfseries\upshape, fontupper=\upshape, blanker, borderline west={4pt}{-8pt}{orange!75!white}},
	asidestyle/.style={enhanced, breakable, beforeafter skip balanced=10pt, blanker, borderline west={4pt}{-8pt}{black!50!white}},
}

\newtcbtheorem[number within=subsection]{example}{Example}{exstyle}{ex}
\newtcbtheorem[number within=subsection]{examples}{Examples}{exstyle}{ex}

\newtcolorbox{aside}{asidestyle}

\begin{document}
\section{The KdV Equation}

\subsection{Preliminaries}
The simplest model of a wave is the one dimensional wave equation:
\begin{equation}
\frac{\partial^2 u}{\partial t^2} - c^2 \frac{\partial^2 u}{\partial x^2} = 0
\end{equation}
with $u(x, t)$ giving the amplitude of the wave and positive constant $c$.
The wave equation's general solution, expressed in terms of \textit{characteristic variables} $(x \pm ct)$, is
\begin{equation}
	u(x, t) = f(x - ct) + g(x + ct)
\end{equation}
where $f, g$ are determined by given initial conditions. This solution is known as \textit{d'Alembert's solution}. From a glance at the solution it should be clear that it describes two waves, one moving to the left, and one to the right, both at speed $c$. The linearity of the wave equation allows for the principle of superposition, and as such neither wave interacts with itself nor one another. 
\begin{aside} {\textit{How does d'Alembert's solution relate to the "Fourier Ansatz" solution to the wave equation?}} {}
\medbreak Using a trig identity, the general solution to the wave equation that arises from the Fourier Ansatz can be turned into d'Alembert's solution (see \href{https://math.libretexts.org/Bookshelves/Differential_Equations/Differential_Equations_(Chasnov)/09%3A_Partial_Differential_Equations/9.06%3A_Solution_of_the_Wave_Equation}{here})
\end{aside}

Let's make an additional simplification, and consider only a wave which propagates in a single direction (say, $g = 0$. Then $u(x, t) = f(x - ct)$ satisfies the equation
\begin{equation}
	u_{t} + cu_{x} = 0
\end{equation}
In this solution, $f$ does not change shape (justification, which I do not fully understand, is given in the text). Of course, this is therefore the simplest wave; if we lessen our simplifying assumptions, and include, say, dispersion, for which there is a new \textit{dispersive} wave equation:
\begin{equation}
	u_t + u_x + u_{xxx} = 0
\end{equation}

We can consider a solution known as the \textit{harmonic} wave solution
\begin{equation}
	u(x, t) = e^{i(kx - \omega t)}
	\label{eq:harmonic sol}
\end{equation}

(This is a simplification of the general solution to the standard wave equation -- see \href{https://github.com/snoopboopsnoop/math-126-notes/blob/main/Physics%20105/Section%208%20notes.pdf}{here})

\tab\eqref{eq:harmonic sol} is a solution to the dispersive wave equation if
\begin{equation}
	\omega = k - k^3
\end{equation}

This is known as a \textit{dispersion relation} (see ibid.) so we can call $k$ the wavenumber and $\omega$ the frequency. Some algebra produces the following equations,
\begin{equation*}
	kx - \omega t = k(x-(1-k^2)t)
\end{equation*}
\begin{equation}
	c = \frac{\omega}{k} = 1 - k^2
\end{equation}

This last equation says that the speed at which the wave propagates depends on its wavenumber (this value, known as the \texit{phase velocity}, describes the velocity of a single component of a wave profile). Thus waves of different wave numbers propagate at different velocities, the defining characteristic of a dispersive wave.\medbreak

We can also define the \textit{group velocity},
\begin{equation}
	c_g = \diff{\omega}{k} = 1 - 3k^2
\end{equation}

which describes the velocity of a whole wave packet. \href{https://www.youtube.com/watch?v=tlM9vq-bepA}{This video} is a great visualization of the difference between group and phase velocity. For many wave motions it is the case that $c_g \leq c$, though the video above is one such exception.

\begin{aside}
	Can a wave be described as two components like \eqref{eq:harmonic sol} but with different wavenumbers? That is, can a single wave have two components that move at different velocities?
\end{aside}

The book says yes, we \textit{can} represent a wave as a superposition of different harmonic solutions of varying wavenumbers:
\begin{equation}
	u(x, t) = \int_{-\infty}^{\infty} A(k) e^{i(kx - \omega(k)t)}\, dk
\end{equation}

(Note similarity to Fourier transform?) The effect of this superposition is a wave profile which changes as it moves; the differing speeds of the components cause the wave to \textit{disperse}.



\end{document}
