\documentclass[12pt]{article}

%packages for math formatting and symbols
\usepackage{amssymb}
\usepackage{amsmath}
\usepackage{amsthm}
\usepackage{graphicx}
\usepackage{float}
\usepackage{hyperref}

\graphicspath{{./img/}}

%global lengths/margins
\textheight 22cm
\textwidth 17cm
\oddsidemargin -0.5cm
\evensidemargin -0.5cm
\topmargin -1.5cm
\topskip 0cm
\headheight 0.5cm
\headsep 1cm
\marginparwidth 1.2cm

%new commands
\def\dx{\mathrm{d}x}
\def\Z{\mathbb{Z}}
\def\N{\mathbb{N}}
\def\R{\mathbb{R}}
\def\Q{\mathbb{Q}}
\def\C{\mathbb{C}}
\def\qedsymbol{$\blacksquare$}

%theorems
\newtheorem{theorem}{Theorem}
\newtheorem*{theorem*}{Theorem}
\newtheorem{lemma}{Lemma}
\newtheorem*{lemma*}{Lemma}
\newtheorem{prop}{Proposition}
\newtheorem*{prop*}{Proposition}


\begin{document}
\section{The KdV Equation}

\subsection{Preliminaries}
The simplest model of a wave is the one dimensional wave equation:
\begin{equation}
\frac{\partial^2 u}{\partial t^2} - c^2 \frac{\partial^2 u}{\partial x^2} = 0
\end{equation}
with $u(x, t)$ giving the amplitude of the wave and positive constant $c$.
The wave equation's general solution, expressed in terms of \textit{characteristic variables} $(x \pm ct)$, is
\begin{equation}
	u(x, t) = f(x - ct) + g(x + ct)
\end{equation}
where $f, g$ are determined by given initial conditions. This solution is known as \textit{d'Alembert's solution}. From a glance at the solution it should be clear that it describes two waves, one moving to the left, and one to the right, both at speed $c$. The linearity of the wave equation allows for the principle of superposition, and as such neither wave interacts with itself nor one another. 
\subsubsection {\textit{How does d'Alembert's solution relate to the "Fourier Ansatz" solution to the wave equation?}}
Using a trig identity, the general solution to the wave equation that arises from the Fourier Ansatz can be turned into d'Alembert's solution (see \href{https://math.libretexts.org/Bookshelves/Differential_Equations/Differential_Equations_(Chasnov)/09%3A_Partial_Differential_Equations/9.06%3A_Solution_of_the_Wave_Equation}{here}) 
Let's make an additional simplification, and consider only a wave which propagates in a single direction (say, $g = 0$. Then $u(x, t) = f(x - ct)$ satisfies the equation
\begin{equation}
	u_{t} + cu_{x} = 0
\end{equation}
In this solution, $f$ does not change shape (justification, which I do not fully understand, is given in the text). Of course, this is therefore the simplest wave; if we lessen our simplifying assumptions, and include, say, dispersion, for which there is a new \textit{dispersive} wave equation:
\begin{equation}
	u_t + u_x + u_{xxx} = 0
\end{equation}



\end{document}
