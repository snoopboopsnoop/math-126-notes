\documentclass[12pt]{article}

%packages for math formatting and symbols
\usepackage{amssymb}
\usepackage{amsmath}
\usepackage{amsthm}
\usepackage{graphicx}
\usepackage{float}
\usepackage{hyperref}
\usepackage[breakable, theorems, skins]{tcolorbox}

\graphicspath{{./img/}}

%global lengths/margins
\allowdisplaybreaks
\unitlength 1cm
\textheight 22cm
\textwidth 17cm
\oddsidemargin -0.5cm
\evensidemargin -0.5cm
\topmargin -1.5cm
\topskip 0cm
\headheight 0.5cm
\headsep 1cm
%\marginparwidth 1.2cm
\newlength\doubleind
\addtolength{\doubleind}{\leftmargini}
\addtolength{\doubleind}{\leftmarginii}
\parindent 0pt
\def\lstsp{\hspace{\labelsep}}


%new commands
\def\dx{\mathrm{d}x}
\def\Z{\mathbb{Z}}
\def\N{\mathbb{N}}
\def\R{\mathbb{R}}
\def\Q{\mathbb{Q}}
\def\C{\mathbb{C}}
\def\qedsymbol{$\blacksquare$}

%theorems
\newtheorem{theorem}{Theorem}
\newtheorem*{theorem*}{Theorem}
\newtheorem{lemma}{Lemma}
\newtheorem*{lemma*}{Lemma}
\newtheorem{prop}{Proposition}
\newtheorem*{prop*}{Proposition}

\tcbset{
	exstyle/.style={enhanced, breakable, beforeafter skip balanced=10pt, coltitle=black, theorem style=plain, terminator sign={.\ \ \ }, fonttitle=\bfseries\upshape, fontupper=\upshape, blanker, borderline west={4pt}{-8pt}{orange!75!white}},
}

\newtcbtheorem[number within=subsubsection]{example}{Example}{exstyle}{ex}
\newtcbtheorem[number within=subsubsection]{examples}{Examples}{exstyle}{ex}

\begin{document}
\setcounter{section}{8}
\section{Continuum Mechanics}
\par\noindent\rule{\textwidth}{1pt}
\subsection{Example: Motion of a String}
We have an elastic string lying along the x-axis. Consider when we displace the string in the $y$ direction by $y(x)$. If we slice the string into infinitesimally small segments of mass $m$, length $l$, and tension $T$, balancing the forces gives (using small angle approximation)

\begin{equation}
	\begin{aligned}
		m\ddot{y}_n &= -\frac{T}{l}(y_n - y_{n-1}) + \frac{T}{l}(y_{n+1} - y_n)\\
		\ddot{y}_n &= \frac{T}{ml}(y_{n+1} - 2y_n + y_{n-1})
	\end{aligned}
\end{equation}

Taking the limit as $m, l \to 0$ and $\rho = \frac{m}{l}$ fixed




\end{document}
